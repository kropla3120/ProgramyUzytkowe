\documentclass[12pt, letterpaper, titlepage]{article}
\usepackage[left=2.5cm, right=2.5cm, top=2.5cm, bottom=2.5cm]{geometry}
\usepackage[MeX]{polski}
\usepackage[utf8]{inputenc}
\usepackage{graphicx}
\usepackage{enumerate}
\usepackage{amsmath} %pakiet matematyczny
\usepackage{amssymb} %pakiet dodatkowych symboli
\title{Kolokwium pierwsze}
\author{Mateusz Kroplewski}
\date{06.12.2022}
\begin{document}
\maketitle
\section{Szyfrowanie blokowe}
\textbf{Szyfrowanie blokowe} \cite{blokowe} traktuje poszczególne bloki tekstu jawnego jako odrębne całości i każdy z nich produkuje szyfrogram tej samej długości. Zazwyczaj używane są bloki o długości 64 lub 128 bitów. Podobnie jak w przypadku szyfru strumieniowego obaj uczestnicy komunikacji  współdzielą ten sam klucz. Przy wykorzystaniu różnych trybów operacyjnych,za pomocą szyfru blokowego można osiągnąć efekty podobne do tych , jakie daje szyfrowanie strumieniowe. \\ Wiele wysiłku poświęcono zbadaniu właściwości szyfrów blokowych. Generalnie znajdują one wyraźnie większe zastosowanie niż szyfry strumieniowe: większość aplikacji sieciowych realizujących szyfrowanie symetryczne wykorzystuje właśnie szyfry blokowe.
\subsection{Standard DES}
Najbardziej rozpowszechnionym obecnie schematem szyfrowania jest DES (\textit{Data Encryption Standard}) \cite{des}, przyjęty w 1977 roku przez Narodowe Biuro Standaryzacji USA (\textit{National Bureau of Standards}), obecnie Narodowy Instytut Normalizacji i Technologi (NIST -\textit{National Institute of Standards and Technology}) jako Federalny Standard Przetwarzania Informacji nr46 (FIST PUB 46). Sam algorytm szyfrowania określany jest jako \textit{Data Encryption Algorithm},w skrócie DEA. W standardzie DES dane wejściowe przetwarzane są w 64-bitowych blokach przy użyciu 56-bitowego klucza: algorytm transformuje 64-bitowe bloki tekstu jawnego na64-bitowe bloki szyfrogramu, te same kroki i przy użyciu tego samego klucza wykonywane są w ramach deszyfracji.
\subsection{Szyfrowanie w standardzie DES}
Tak jak w każdym schemacie szyfrowania informację wejściową stanowią dwa elementy: tekst
jawny i klucz. Tekst jawny ma postać 64-bitowego bloku, długość klucza wynosi 56 bitów.
Przetwarzanie tekstu jawnego odbywa się w trzech fazach. Pierwszą z nich stanowi \textbf{permutacja wstępna} (IP - \textit{initial permutation}), w ramach której wejściowy blok 64-bitowy
przekształcany jest do postaci permutowanego wejścia (\textit{permuted input}). Faza druga stano-
wi ciąg \textbf{16 jednakowych rund}, z których każda obejmuje permutowanie i podstawianie,
32-bitowe połówki bloku stanowiącego wynik ostatniej rundy zamieniane są miejscami, po
czym blok ten (zwany \textbf{wyjściem wstępnym}-\textit{preoutput}) poddawany jest fazie trzeciej, czyli
\textbf{permutacji stanowiącej odwrotność permutacji wstępnej}
\\ \\
Druga część odzwierciedla natomiast ciąg przekształceń, jakim poddawany jest 56-bitowy
klucz. Pierwszym z tych przekształceń jest permutacja początkowa. Następnie w ramach
każdej z 16 rund produkowane są \textit{podklucze} $K_i$. Produkcja ta odbywa się w każdej rundzie dwuetapowo. W pierwszym etapie otrzymany podklucz poddawany jest operacji lewostronnego obrotu (o 1 lub 2 bity), wynik tego etapu przekazywany jest do następnej rundy; jest on
jednocześnie permutowany (to drugi etap - wybór permutowany 2), dając w rezultacie klucz
$K_i$. Permutacja wykonywana w drugim etapie jest identyczna dla każdej rundy, jednakże ze
względu na powtarzaną (w pierwszym etapie) operację lewostronnego obrotu kolejne klucze  
$K_i$ różnią się od siebie. \\\\
Przykład permutacji wstępnej w tabeli 1.2.
\begin{table}[h]
\centering\caption{Permutacja wstępna IP}
\begin{tabular}{|cccccccc|}
\hline
58&50&42&34&26&18&10&2\\
60&52&44&36&28&20&12&4\\
62&54&46&38&30&22&14&6\\
64&56&48&40&32&24&16&8\\
57&49&41&33&25&17&9&1\\
59&51&43&35&27&19&11&3\\
61&53&45&37&29&21&13&5\\
63&55&47&39&31&23&15&7\\
\hline
\end{tabular}
\end{table}
\subsection{Szczegóły jednej rundy}
Oznaczmy przez L i R (odpowiednio) lewy i prawy półblok 64-biotowego bloku wejściowego;
identycznie jak w klasycznej wersji szyfru Feistela są one traktowane oddzielnie, według formuły
\begin{center}
\begin{tabular}{l}
$L_i = R_{i-1}$ \\
$R_i = L_{i-1} \ \bigoplus \	 F(R_{i-1}, K_i)$  
\end{tabular}
\end{center}
Klucz rundy $K_i$ ma rozmiar 48 bitów, wejściowy półblok R - 32 bity. Półblok ten jest wpierw
rozszerzany do 48 bitów za pomocą złożenia permutacji i dublowania wybranych 16 bitów.
Wynik tej operacji składany jest z kluczem $K_i$ przez operację XOR. Wynik złożenia przetwarzany jest przez funkcję podstawieniową F zwracającą wartość 32-bitową, która następnie
jest permutowana.
\newpage
\section{Ciasto jogurtowe z żurawiną}
Fantastyczne, wilgotne, mięciutkie ciasto jogurtowe z żurawiną, suto polukrowane. Latem
piekłam to ciasto z dodatkiem agrestu, teraz występuje w zimowej odsłonie. Ciasto długo
utrzymuje świeżość a owoce żurawiny pasują tutaj jak żadne inne! \cite{ciasto}
\subsection*{Składniki na ciasto jogurtowe z żurawiną}
\begin{enumerate}
\item 165g masła
\item 160 g drobnego cukru do wypieków
\item 8 g cukru wanilinowego lub 1 łyżeczka ekstraktu z wanilii
\item 3 duże jajka
\item 150 g jogurtu naturalnego lub greckiego
\item 270 g mąki pszennej
\item 2 łyżeczki proszku do pieczenia
\item 300 g żurawiny
\end{enumerate}
Wszystkie składniki powinny być w temperaturze pokojowej. \\
\\
W misie miksera umieścić masło i oba cukry (lub cukier i wanilię). Utrzeć do powstania
jasnej i puszystej masy maślanej. Dodawać jajka, jedno po drugim, ucierając do całkowite-
go połączenia się składników po każdym dodaniu (ciasto na tym etapie może wyglądać na
zwarzone, ale nie ma to wpływu na wypiek końcowy). Bezpośrednio do utartych składników
przesiać mąkę pszenną i proszek do pieczenia oraz dodać jogurt. Wymieszać szpatułką tylko
do połączenia się składników, nie dłużej. Dodać żurawinę i krótko wymieszać. \\
\\
Formę o średnicy 25 cm wyłożyć papierem do pieczenia. Przełożyć do niej ciasto, wyrównać.\\
\\
Ciasto jogurtowe z żurawiną piec w temperaturze 170ºC, bez termoobiegu, przez około 50
minut lub dłużej, do tzw. suchego patyczka. Wyjąć i wystudzić w formie. Polukrować.
\subsection*{Waniliowy lukier}
\begin{enumerate}[•]
\item 1 szklanka cukru pudru
\item 2 - 3 łyżki wrzącej wody lub soku z cytryny
\item 1 łyżeczka ekstraktu z wanilii 
\end{enumerate}
Wszystkie składniki umieścić miseczce i rozetrzeć grzbietem łyżki do otrzymania gęstego
lukru (gęstość lukru regulować dodatkiem dodatkowego cukru pudru lub wody).
\begin{thebibliography}{9}
\bibitem{blokowe}
William Stallings, \textit{Kryptografia  i  bezpieczeństwo  sieci  komputerowych} Helion, 2012, str.105
\bibitem{des}
William Stallings, \textit{Kryptografia  i  bezpieczeństwo  sieci  komputerowych} Helion, 2012, str.115
\bibitem{ciasto}
https://mojewypieki.com/przepis/ciasto-jogurtowe-z-zurawina
\end{thebibliography}
\end{document}