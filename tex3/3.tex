\documentclass[12pt, a4paper]{article}
\usepackage[left=2.5cm, right=2.5cm, top=2.5cm, bottom=2.5cm]{geometry}
\usepackage[MeX]{polski}
\usepackage[utf8]{inputenc}
\usepackage{graphicx}
\usepackage{enumerate}
\usepackage{amsmath} %pakiet matematyczny
\usepackage{amssymb} %pakiet dodatkowych symboli
\title{Ćwiczenia z LaTeX, programy użytkowe}
\author{Mateusz Kroplewski}
\date{15.11.2022}
\begin{document}
\begin{equation}
 \sqrt{ \frac{2^{n}}{2_n}} \neq \sqrt[\frac{1}{3}]{1+n}
\end{equation}
\begin{equation}
	\frac{2^{k}}{2^{k+2}}
\end{equation}
\begin{equation}
\frac{x^{2}}{2^{(x+2)(x-2)^{3}}}
\end{equation}
\begin{equation}
log_2 2^8 = 8	
\end{equation}
\begin{equation}
\sqrt[3]{e^x - log_2 x}
\end{equation}
\begin{equation}
\lim_{n \rightarrow \infty} \sum_{k=1}^{n} \frac{1}{k^2} = \frac{\pi^2}{6}
\end{equation}
\begin{equation}
\int_{2}^{\infty} \frac{1}{\log_{2}x}dx  = \frac{1}{x}sin x = 1 - cos^{2}(x)
\end{equation}
\begin{equation}
\left[
\begin{array}{cccc}
a_{11} & a_{12} & \ldots & a_{1K} \\
a_{21} & a_{22} & \ldots & a_{2K} \\
\vdots & \vdots & \ddots & \vdots \\
a_{K1} & a_{K2} & \ldots & a_{KK} \\
\end{array}
\right]
*
\left[
\begin{array}{c}
x_1 \\
x_2 \\
\vdots \\
x_K \\
\end{array}
\right]
=
\left[
\begin{array}{c}
b_1 \\
b_2 \\
\vdots \\
b_K \\
\end{array}
\right]
\end{equation}
\begin{equation}
(a_1 = a_{1}(x)) \wedge (a_2 = a_{2}(x)) \wedge \ldots \wedge (a_k = a_{k}(x)) \Rightarrow (d = d(u))
\end{equation}
\begin{equation}
[x]_{A} = \lbrace y \in U : a(x) = a(y), \forall a \in A \rbrace \textrm{, where the control object} x \in U
\end{equation}
\begin{equation}
T : [0,1] \times [0,1] \rightarrow [0,1]
\end{equation}
\begin{equation}
\lim_{x \rightarrow \infty} \exp(-x) = 0
\end{equation}
\begin{equation}
\frac{n!}{k!(n-k)!} = {n \choose k}
\end{equation}
\begin{equation}
P \left( A=2 \bigg| \frac{A^2}{B} > 4 \right)
\end{equation}
\begin{equation}
S^{C_i}(a) = \frac{(\bar{C}^{a}_i) - (\hat{C}^{a}_i)^2}{Z_{\bar{C}^{a^2}_i} + Z_{\hat{C}^{a^2}_i}}, a \in A
\end{equation}
\end{document}
